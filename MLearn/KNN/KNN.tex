\documentclass[12 pt]{article}        	%sets the font to 12 pt and says this is an article (as opposed to book or other documents)
\usepackage{amsfonts, amssymb,hyperref}					% packages to get the fonts, symbols used in most math
\hypersetup{
    colorlinks=true,
    linkcolor=blue,
    filecolor=magenta,      
    urlcolor=cyan,
    pdftitle={Overleaf Example},
    pdfpagemode=FullScreen,
    }
%\usepackage{setspace}               		% Together with \doublespacing below allows for doublespacing of the document

\oddsidemargin=-0.5cm                 	% These three commands create the margins required for class
\setlength{\textwidth}{6.5in}         	%
\addtolength{\voffset}{-20pt}        		%
\addtolength{\headsep}{25pt}           	%



\pagestyle{myheadings}                           	% tells LaTeX to allow you to enter information in the heading
\markright{Pham Quoc Nam\hfill \today \hfill} 	% put your name instead of Murphy Waggoner 
																									% and put the proposition number from the book
                                                	% LaTeX will put your name on the left, the date the paper 
                                                	% is generated in the middle 
                                                 	% and a page number on the right



\newcommand{\eqn}[0]{\begin{array}{rcl}}%begin an aligned equation - allows for aligning = or inequalities.  Always use with $$ $$
\newcommand{\eqnend}[0]{\end{array} }  	%end the aligned equation

\newcommand{\qed}[0]{$\square$}        	% make an unfilled square the default for ending a proof

%\doublespacing                         	% Together with the package setspace above allows for doublespacing of the document

\begin{document}												% end of preamble and beginning of text that will be printed

        																% makes the word Proposition and the proposition number bold face  
\section*{Definition}
\textbf{KNN(K-Nearest Neighbours)} is an algorithm for supervised learning. Where the data is 'trained' with data points corresponding to their classification. Once a point is to be predicted, it takes into account the 'K' nearest points to it to determine it's classification.
KNN is \textbf{\textit{non-parametic}}(means that it doesn't make any assumptions on the underlying data distribution),\textbf{\textit{instance-based}}(means that our algorithm does not implicitly learn a model. Instead, it chooses to memorize the training instances).

\textbf{Instance-based learning} learn the training examples by heart and then generalizes to new instances based on some similarity measure. It is called instance-based because it builds the hypotheses from the training instances. It is also known as memory-based learning or lazy-learning (because they delay processing until a new instance must be classified)
\paragraph*{More info:} \href{https://www.geeksforgeeks.org/instance-based-learning/}{Instance-based-learning}

\section*{How does K-NN aLgorithm work? }
\textbf{KNN} is used for classification -- the output is \underline{class membership}(predicts a class - a discrete value)
\par
\textbf{three main elements:}
\begin{itemize}
	\item A set of label objects
	\item The distance between objects
	\item The value of \textbf{K}
  \end{itemize}

  
\end{document}